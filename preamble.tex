% Packages
\usepackage[danish]{babel} % Dansk sprog
\usepackage{natbib} % Referencer
\bibliographystyle{abbrvnat}
\usepackage{amsmath, amsfonts, amssymb, everysel} % Avanceret matematik
\usepackage{fancyhdr, lastpage} % Sidehoved og sidefod
\setlength{\headheight}{15pt}
\usepackage{lmodern} % Pænere standard skrifttype
\usepackage{lipsum} % Fyldtekst
\usepackage{tcolorbox, colortbl, varwidth} % Boks rundt om tekst
\tcbuselibrary{skins, breakable}
\usepackage{array, tabularx, siunitx} % Tabeller i bokse
\usepackage{fontspec} % Nye skrifttyper
\usepackage{enumitem} % Lettere ændringer i enumerate miljøet
\usepackage{changepage} % Ændre margen
\usepackage{sectsty} % Ændre overskrifter
\usepackage[titles]{tocloft} % Fix indholdsfortegnelse tekst
\usepackage{xcolor}
\usepackage{graphicx} % Billeder
\graphicspath{{images/}}
\usepackage{microtype} % Pænere paragrafer
\usepackage{styles/dtucolors} % DTU farver
\usepackage[colorlinks = true, allcolors = s13, linkcolor = black]{hyperref} % Hyperlinks i dokumentet
    
% Sans-serif skrifttype
\setsansfont[
    Ligatures=TeX,
    Extension=.otf,
    UprightFont=*-regular,
    BoldFont=*-bold,
    ItalicFont=*-italic,
    BoldItalicFont=*-bolditalic,
    Scale=0.9
]{texgyreadventor}

% Sidehoved og sidefod generelt
\pagestyle{fancy}
    \fancyhf{}
    \lhead{\footnotesize\sffamily\projecttitle~ --~ \nouppercase{\leftmark}}
    \rhead{\footnotesize\sffamily\projectdate}
    \lfoot{\footnotesize\sffamily\projectauthor}
    \rfoot{\footnotesize\sffamily Side \thepage~af \pageref{LastPage}}
    \renewcommand{\footrulewidth}{0.5pt}

% Sidehoved og sidefod til indholdsfortegnelsen
\fancypagestyle{toc}
{
    \fancyhf{}
    \lhead{\footnotesize\sffamily\projecttitle}
    \rhead{\footnotesize\sffamily\projectdate}
    \lfoot{\footnotesize\sffamily\projectauthor}
    \rfoot{\footnotesize\sffamily Indhold}
    \renewcommand{\footrulewidth}{0.5pt}
    \setcounter{page}{0}
}

% Overskrifter
\sectionfont{\raggedright\LARGE\normalfont\sffamily}
\subsectionfont{\large\normalfont\sffamily}
\subsubsectionfont{\it\sffamily}
\renewcommand{\thesection}{\hspace{-1em}}
\renewcommand{\thesubsection}{\hspace{-1em}}
\renewcommand{\thesubsubsection}{\hspace{-1em}}

% Boks-indstillinger
\tcbset{enhanced, skin = bicolor, breakable,
colframe = dtured, coltitle = white, colupper = white, colback = dtured, colbacklower = white,
fonttitle = \sffamily\bfseries\large, nobeforeafter,
toptitle = 1mm, bottomtitle = 1mm, after title = \vspace{2mm}\hrule\vspace{-1mm}}

% Fancy itemize
\newenvironment{myitemize}{%
\begin{itemize}}{\end{itemize}}
\tcolorboxenvironment{myitemize}{blanker, before skip=6pt, after skip=6pt, borderline west={2mm}{0pt}{dtured}, colupper = black, left = -2mm}

% Fancy enumerate
\newenvironment{myenum}{%
\begin{enumerate}}{\end{enumerate}}
\tcolorboxenvironment{myenum}{blanker, before skip=6pt, after skip=6pt, borderline west={2mm}{0pt}{dtured}, colupper = black, left = -1mm}

% Fancy equation
\newenvironment{myeq}{%
\begin{equation}}{\end{equation}}
\tcolorboxenvironment{myeq}{blanker, before skip=6pt, after skip=6pt, borderline west={2mm}{0pt}{dtured}, colupper = black, left = 4mm}
\newenvironment{myeq*}{%
\begin{equation*}}{\end{equation*}}
\tcolorboxenvironment{myeq*}{blanker, before skip=6pt, after skip=6pt, borderline west={2mm}{0pt}{dtured}, colupper = black, left = 4mm}

% Fancy align
\newenvironment{myalign}{%
\begin{align}}{\end{align}}
\tcolorboxenvironment{myalign}{blanker, before skip=6pt, after skip=6pt, borderline west={2mm}{0pt}{dtured}, colupper = black, left = 4mm}
\newenvironment{myalign*}{%
\begin{align*}}{\end{align*}}
\tcolorboxenvironment{myalign*}{blanker, before skip=6pt, after skip=6pt, borderline west={2mm}{0pt}{dtured}, colupper = black, left = 4mm}

% % Fancy text
\newenvironment{mytext}{\vspace{1mm}}{\vspace{.5mm}}
\tcolorboxenvironment{mytext}{blanker, before skip=6pt, after skip=6pt, borderline west={2mm}{0pt}{dtured}, colupper = black, left = 4mm}

% Fix kolonner i tabeller
\newcolumntype{L}[1]{>{\raggedright\let\newline\\\arraybackslash\hspace{0pt}}m{#1}}
\newcolumntype{C}[1]{>{\centering\let\newline\\\arraybackslash\hspace{0pt}}m{#1}}
\newcolumntype{R}[1]{>{\raggedleft\let\newline\\\arraybackslash\hspace{0pt}}m{#1}}

% Vector command
\renewcommand{\vec}[1]{
\begin{bmatrix} #1 \end{bmatrix}}
